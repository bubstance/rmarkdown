% Options for packages loaded elsewhere
\PassOptionsToPackage{unicode}{hyperref}
\PassOptionsToPackage{hyphens}{url}
%
\documentclass[
]{article}
\usepackage{amsmath,amssymb}
\usepackage{lmodern}
\usepackage{iftex}
\ifPDFTeX
  \usepackage[T1]{fontenc}
  \usepackage[utf8]{inputenc}
  \usepackage{textcomp} % provide euro and other symbols
\else % if luatex or xetex
  \usepackage{unicode-math}
  \defaultfontfeatures{Scale=MatchLowercase}
  \defaultfontfeatures[\rmfamily]{Ligatures=TeX,Scale=1}
\fi
% Use upquote if available, for straight quotes in verbatim environments
\IfFileExists{upquote.sty}{\usepackage{upquote}}{}
\IfFileExists{microtype.sty}{% use microtype if available
  \usepackage[]{microtype}
  \UseMicrotypeSet[protrusion]{basicmath} % disable protrusion for tt fonts
}{}
\makeatletter
\@ifundefined{KOMAClassName}{% if non-KOMA class
  \IfFileExists{parskip.sty}{%
    \usepackage{parskip}
  }{% else
    \setlength{\parindent}{0pt}
    \setlength{\parskip}{6pt plus 2pt minus 1pt}}
}{% if KOMA class
  \KOMAoptions{parskip=half}}
\makeatother
\usepackage{xcolor}
\usepackage[margin=1in]{geometry}
\usepackage{color}
\usepackage{fancyvrb}
\newcommand{\VerbBar}{|}
\newcommand{\VERB}{\Verb[commandchars=\\\{\}]}
\DefineVerbatimEnvironment{Highlighting}{Verbatim}{commandchars=\\\{\}}
% Add ',fontsize=\small' for more characters per line
\usepackage{framed}
\definecolor{shadecolor}{RGB}{248,248,248}
\newenvironment{Shaded}{\begin{snugshade}}{\end{snugshade}}
\newcommand{\AlertTok}[1]{\textcolor[rgb]{0.94,0.16,0.16}{#1}}
\newcommand{\AnnotationTok}[1]{\textcolor[rgb]{0.56,0.35,0.01}{\textbf{\textit{#1}}}}
\newcommand{\AttributeTok}[1]{\textcolor[rgb]{0.77,0.63,0.00}{#1}}
\newcommand{\BaseNTok}[1]{\textcolor[rgb]{0.00,0.00,0.81}{#1}}
\newcommand{\BuiltInTok}[1]{#1}
\newcommand{\CharTok}[1]{\textcolor[rgb]{0.31,0.60,0.02}{#1}}
\newcommand{\CommentTok}[1]{\textcolor[rgb]{0.56,0.35,0.01}{\textit{#1}}}
\newcommand{\CommentVarTok}[1]{\textcolor[rgb]{0.56,0.35,0.01}{\textbf{\textit{#1}}}}
\newcommand{\ConstantTok}[1]{\textcolor[rgb]{0.00,0.00,0.00}{#1}}
\newcommand{\ControlFlowTok}[1]{\textcolor[rgb]{0.13,0.29,0.53}{\textbf{#1}}}
\newcommand{\DataTypeTok}[1]{\textcolor[rgb]{0.13,0.29,0.53}{#1}}
\newcommand{\DecValTok}[1]{\textcolor[rgb]{0.00,0.00,0.81}{#1}}
\newcommand{\DocumentationTok}[1]{\textcolor[rgb]{0.56,0.35,0.01}{\textbf{\textit{#1}}}}
\newcommand{\ErrorTok}[1]{\textcolor[rgb]{0.64,0.00,0.00}{\textbf{#1}}}
\newcommand{\ExtensionTok}[1]{#1}
\newcommand{\FloatTok}[1]{\textcolor[rgb]{0.00,0.00,0.81}{#1}}
\newcommand{\FunctionTok}[1]{\textcolor[rgb]{0.00,0.00,0.00}{#1}}
\newcommand{\ImportTok}[1]{#1}
\newcommand{\InformationTok}[1]{\textcolor[rgb]{0.56,0.35,0.01}{\textbf{\textit{#1}}}}
\newcommand{\KeywordTok}[1]{\textcolor[rgb]{0.13,0.29,0.53}{\textbf{#1}}}
\newcommand{\NormalTok}[1]{#1}
\newcommand{\OperatorTok}[1]{\textcolor[rgb]{0.81,0.36,0.00}{\textbf{#1}}}
\newcommand{\OtherTok}[1]{\textcolor[rgb]{0.56,0.35,0.01}{#1}}
\newcommand{\PreprocessorTok}[1]{\textcolor[rgb]{0.56,0.35,0.01}{\textit{#1}}}
\newcommand{\RegionMarkerTok}[1]{#1}
\newcommand{\SpecialCharTok}[1]{\textcolor[rgb]{0.00,0.00,0.00}{#1}}
\newcommand{\SpecialStringTok}[1]{\textcolor[rgb]{0.31,0.60,0.02}{#1}}
\newcommand{\StringTok}[1]{\textcolor[rgb]{0.31,0.60,0.02}{#1}}
\newcommand{\VariableTok}[1]{\textcolor[rgb]{0.00,0.00,0.00}{#1}}
\newcommand{\VerbatimStringTok}[1]{\textcolor[rgb]{0.31,0.60,0.02}{#1}}
\newcommand{\WarningTok}[1]{\textcolor[rgb]{0.56,0.35,0.01}{\textbf{\textit{#1}}}}
\usepackage{longtable,booktabs,array}
\usepackage{calc} % for calculating minipage widths
% Correct order of tables after \paragraph or \subparagraph
\usepackage{etoolbox}
\makeatletter
\patchcmd\longtable{\par}{\if@noskipsec\mbox{}\fi\par}{}{}
\makeatother
% Allow footnotes in longtable head/foot
\IfFileExists{footnotehyper.sty}{\usepackage{footnotehyper}}{\usepackage{footnote}}
\makesavenoteenv{longtable}
\usepackage{graphicx}
\makeatletter
\def\maxwidth{\ifdim\Gin@nat@width>\linewidth\linewidth\else\Gin@nat@width\fi}
\def\maxheight{\ifdim\Gin@nat@height>\textheight\textheight\else\Gin@nat@height\fi}
\makeatother
% Scale images if necessary, so that they will not overflow the page
% margins by default, and it is still possible to overwrite the defaults
% using explicit options in \includegraphics[width, height, ...]{}
\setkeys{Gin}{width=\maxwidth,height=\maxheight,keepaspectratio}
% Set default figure placement to htbp
\makeatletter
\def\fps@figure{htbp}
\makeatother
\setlength{\emergencystretch}{3em} % prevent overfull lines
\providecommand{\tightlist}{%
  \setlength{\itemsep}{0pt}\setlength{\parskip}{0pt}}
\setcounter{secnumdepth}{5}
\newlength{\cslhangindent}
\setlength{\cslhangindent}{1.5em}
\newlength{\csllabelwidth}
\setlength{\csllabelwidth}{3em}
\newlength{\cslentryspacingunit} % times entry-spacing
\setlength{\cslentryspacingunit}{\parskip}
\newenvironment{CSLReferences}[2] % #1 hanging-ident, #2 entry spacing
 {% don't indent paragraphs
  \setlength{\parindent}{0pt}
  % turn on hanging indent if param 1 is 1
  \ifodd #1
  \let\oldpar\par
  \def\par{\hangindent=\cslhangindent\oldpar}
  \fi
  % set entry spacing
  \setlength{\parskip}{#2\cslentryspacingunit}
 }%
 {}
\usepackage{calc}
\newcommand{\CSLBlock}[1]{#1\hfill\break}
\newcommand{\CSLLeftMargin}[1]{\parbox[t]{\csllabelwidth}{#1}}
\newcommand{\CSLRightInline}[1]{\parbox[t]{\linewidth - \csllabelwidth}{#1}\break}
\newcommand{\CSLIndent}[1]{\hspace{\cslhangindent}#1}
\usepackage{booktabs}
\usepackage{pdfpages}
\usepackage{titling} \renewcommand\maketitlehooka{\null\mbox{}\vfill} \renewcommand\maketitlehookd{\vfill\null}
\usepackage{lipsum}
\usepackage{appendix}
\pagenumbering{gobble}
\ifLuaTeX
  \usepackage{selnolig}  % disable illegal ligatures
\fi
\IfFileExists{bookmark.sty}{\usepackage{bookmark}}{\usepackage{hyperref}}
\IfFileExists{xurl.sty}{\usepackage{xurl}}{} % add URL line breaks if available
\urlstyle{same} % disable monospaced font for URLs
\hypersetup{
  pdfauthor={Kelly G. Roberts },
  hidelinks,
  pdfcreator={LaTeX via pandoc}}

\title{\huge \textsc{Test Document}}
\usepackage{etoolbox}
\makeatletter
\providecommand{\subtitle}[1]{% add subtitle to \maketitle
  \apptocmd{\@title}{\par {\large #1 \par}}{}{}
}
\makeatother
\subtitle{\textit{a small sample of \textbf{R Markdown}}}
\author{Kelly G. Roberts \textit{(kgr)}}
\date{\today}

\begin{document}
\maketitle

\newpage
\pagenumbering{arabic}
\tableofcontents
\newpage

\hypertarget{overview}{%
\section{Overview}\label{overview}}

\section{Introduction}

This article serves as a brief overview of the \LaTeX{} document
preparation system coupled with examples of the complex formatting it
can help achieve. It must be emphasized that this is only a basic guide
to help familiarize you with the power of using \LaTeX{} as a means to
prepare professional-level documentation.

\section{On the Subject of \LaTeX}

\subsection{Okay... so what can I, like, \textit{do} with it?}

Honestly? Whatever you want!

You can do basically everything you could ever need to do when it comes
to formatting a document in a professional and consistent way. This is
one of the key benefits of using \LaTeX{}; as long as you give the
appropriate commands and syntax, your document will appear exactly as
you specified. This is especially true when it comes to formatting items
like equations. Have you ever tried to write a mathematical formula in
something like Word? It can be an absolute nightmare to get the
formatting correct without destroying the other text in the document.

Here's an example of a continued fraction, something that would be
\emph{exceptionally} difficult to write out in Word:

\[ e=2+\frac{1}{1+\frac{1}{2+\frac{2}{3+\frac{3}{4+\frac{4}{5+\ddots}}}}} \]

Now, that may look like it would be complicated to input into your text
editor, but I can assure you that \LaTeX{} makes this quite easy; all it
takes is learning some basic commands.

\section{Examples}

Here's where some examples would be\ldots{}

IF I HAD THEM.

Oh wait\ldots{} I \emph{do} have some. How about block quotes?

\begin{quote}
``If it weren't for my lawyer, I'd still be in prison. It went a lot
faster with two people digging.''

\hfill --- Joe Martin
\end{quote}

\begin{quote}
``I will, in fact, claim that the difference between a bad programmer
and a good one is whether he considers his code or his data structures
more important. Bad programmers worry about the code. Good programmers
worry about data structures and their relationships.''

\hfill --- Linus Torvalds
\end{quote}

This time we actually used another package that we grabbed within our
root \emph{R} environment (\texttt{sudo\ R}) followed by the subcommand
\texttt{install.packages(\textquotesingle{}tufte\textquotesingle{})}. It
allows us to put neat little footers into our block quotes. Neat, huh?

\section{Citing a Bibliography or Reference Document}

As you can see, all of the information that will end up in our
bibliography is here, including the names of editors, the publisher, and
the year of publication. Using this format, we can now call our new
entry by its name, which in this case is \texttt{@aissen99}. Names can
be whatever you choose that is easy to remember for you. I chose to use
the author's last name and the year of publication.

\hypertarget{look-at-this-table}{%
\section{Look at this table}\label{look-at-this-table}}

\begin{longtable}[]{@{}lrrrr@{}}
\toprule()
Model & MPG & Cyl & Disp & HP \\
\midrule()
\endhead
Mazda RX4 & 21.0 & 6 & 160 & 110 \\
Mazda RX4 Wag & 21.0 & 6 & 160 & 110 \\
Datsun 710 & 22.8 & 4 & 108 & 93 \\
Hornet 4 Drive & 21.4 & 6 & 258 & 110 \\
Hornet Sportabout & 18.7 & 8 & 360 & 175 \\
Valiant & 18.1 & 6 & 225 & 105 \\
\bottomrule()
\end{longtable}

\hypertarget{lists-natures-organizer}{%
\subsection{Lists: Nature's Organizer}\label{lists-natures-organizer}}

Boy howdy, that sure is a subsection.

\begin{enumerate}
\def\labelenumi{\arabic{enumi}.}
\setcounter{enumi}{-1}
\tightlist
\item
  Look
\item
  at
\item
  this
\item
  list
\item
  isn't
\item
  it
\item
  neat
\end{enumerate}

\begin{itemize}
\tightlist
\item
  unordered
\item
  this
\item
  time
\end{itemize}

\hypertarget{section-2-electric-boogaloo}{%
\section{Section 2: Electric
Boogaloo}\label{section-2-electric-boogaloo}}

Oh look, some more text.

Very nice.

Remember that table from earlier? Here's a neat code block:

\begin{verbatim}
|Model             |MPG  |Cyl |Disp|HP  |
|:-----------------|----:|---:|---:|---:|
|Mazda RX4         |21.0 |6   |160 |110 |
|Mazda RX4 Wag     |21.0 |6   |160 |110 |
|Datsun 710        |22.8 |4   |108 |93  |
|Hornet 4 Drive    |21.4 |6   |258 |110 |
|Hornet Sportabout |18.7 |8   |360 |175 |
|Valiant           |18.1 |6   |225 |105 |
|Sample            |19.0 |3   |256 |420 |
\end{verbatim}

That's exactly how it looked in this \texttt{.rmd} file.

Pretty fuckin' wild, eh?

\hypertarget{how-about-that-r-though}{%
\subsection{\texorpdfstring{How about that \textbf{R},
though?}{How about that R, though?}}\label{how-about-that-r-though}}

Is it neat? Yes!

Here's something cool that you can do with code blocks: evaluate R code.

\begin{Shaded}
\begin{Highlighting}[]
\DecValTok{2}\SpecialCharTok{+}\DecValTok{2}
\end{Highlighting}
\end{Shaded}

\begin{verbatim}
## [1] 4
\end{verbatim}

In this case, we evaluated the chunk \texttt{2+2} into the output shown
above.

How about some more complicated equations?

\begin{Shaded}
\begin{Highlighting}[]
\DecValTok{5}\SpecialCharTok{*}\DecValTok{40}
\end{Highlighting}
\end{Shaded}

\begin{verbatim}
## [1] 200
\end{verbatim}

\begin{Shaded}
\begin{Highlighting}[]
\DecValTok{3}\SpecialCharTok{\^{}}\DecValTok{9}
\end{Highlighting}
\end{Shaded}

\begin{verbatim}
## [1] 19683
\end{verbatim}

\begin{Shaded}
\begin{Highlighting}[]
\NormalTok{values }\OtherTok{\textless{}{-}} \FunctionTok{rnorm}\NormalTok{(}\DecValTok{5}\NormalTok{)}
\NormalTok{values}
\end{Highlighting}
\end{Shaded}

\begin{verbatim}
## [1] -1.84320935  0.01164040  0.03902493 -0.63100984  0.26086364
\end{verbatim}

Wow, we evaluated the \emph{shit} out of those expressions.

\hypertarget{more-lists}{%
\subsection{More Lists}\label{more-lists}}

Perhaps another list?

\begin{enumerate}
\def\labelenumi{\arabic{enumi}.}
\tightlist
\item
  First

  \begin{itemize}
  \tightlist
  \item
    subfirst
  \item
    submore
  \item
    subagain
  \end{itemize}
\item
  Second

  \begin{itemize}
  \tightlist
  \item
    subsecond

    \begin{itemize}
    \tightlist
    \item
      deeper

      \begin{itemize}
      \tightlist
      \item
        still
      \end{itemize}
    \end{itemize}
  \item
    subsecondsecond
  \end{itemize}
\item
  Third
\end{enumerate}

As stated by Aissen (1999), blah blah blah words words words.

\newpage

\hypertarget{references}{%
\section*{References}\label{references}}
\addcontentsline{toc}{section}{References}

\hypertarget{refs}{}
\begin{CSLReferences}{1}{0}
\leavevmode\vadjust pre{\hypertarget{ref-aissen99}{}}%
Aissen, Judith. 1999. {``External Possessor and Logical Subject in
Tz'utujil.''} In \emph{External Possession}, edited by Doris Payne and
Immanuel Barshi. John Benjamins Publishing Company.

\end{CSLReferences}

\end{document}
